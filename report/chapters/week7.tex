\section*{Week 7: Experimenting with Llama 2 \& different training technqiues}
\addcontentsline{toc}{section}{Week7: Experimenting with Llama 2 \& different training technqiues}

The main goals achieved this week are:
\begin{itemize}[topsep=0pt]
    \item \textbf{7.1} Retraining model5 from the previous week using a smaller training set and larger test set
    \item \textbf{7.2} Found a \href{https://arxiv.org/pdf/2210.07316.pdf}{Massive Text Embedding Benchmark (MTEB)} model to try
    \item \textbf{7.3} One-vs-the-Rest (OvR) multiclass strategy
    \item \textbf{7.4} Prompt engineering results on Llama 2
    \item \textbf{7.5} Reformatted the original Jupyter files (.ipynb) into Python files (.py) for easier integration with Docker containers
    \item \textbf{7.6} Findings from \href{https://github.com/mrwadams/stride-gpt?tab=readme-ov-file#usage}{stride-gpt} on Github
\end{itemize}

\subsection*{7.1 Training model5 on smaller training set $\Longrightarrow$ model6}
I did some research and found out that training models on a smaller training set and larger test set might be useful when training resources are limited.
Sure enough, the attempts were not all futile. \\\\
I noticed a striking pattern between `Information disclosure' (I) and `Elevation of privilege' (E). Most of the time the model either classfies most of the data correctly as the former or latter. This means that if a lot of `I' is classfied correctly, `E' is classified wrongly as `I', vice versa. \\\\
Strangely, this could suggest the keywords for both categories are very similar. The following result was the best that I could achieve with the same architecture as model5. \\

\includegraphics*[scale=0.528]{cmatrix_model6.png}

I believe that the model architecture might be too straightforward. As such, I have implemented a more complex model in the next section.

\subsection*{7.2 More complex model architecture}
This model is based on the \href{https://arxiv.org/pdf/2210.07316.pdf}{Massive Text Embedding Benchmark (MTEB)} model. The model is a transformer-based model. The paper suggests that transformers might help in injecting context awareness into the language model via self-attention.

\subsection*{7.3 One-vs-the-Rest (OvR) multiclass strategy}
This OvR strategy is a way to break down a mutliclass classification problem into mutliple binary classification problems, effectively ``training a separate models for each category and combining them into a single model''. \\
For example, if there are 3 categories red, blue and green, the OvR strategy would train 3 models: \\
\begin{align*}
    \text{Model 1:} \quad & \text{red vs. not red} \\
    \text{Model 2:} \quad & \text{blue vs. not blue} \\
    \text{Model 3:} \quad & \text{green vs. not green} \\
\end{align*}
However, the results obtained were not favourable.