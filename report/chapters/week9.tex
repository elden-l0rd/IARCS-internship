\section*{Week 9: Augmenting existing data}
\addcontentsline{toc}{section}{Week 9: Augmenting existing data}

\textbf{Taking another look at the training data:} \\
Other than trying out prompt engineering on Llama 2, I believe there should be another way for training a model to classify the STRIDE categories.
There should definitely be an (easier) way which just might not be that straightforward.
The most significant problem is that I do not have enough training data. Upon further research, I found that back-translation is a way to augment more data for NLP. This might not guarantee an improvement in the model's performance but it is worth a try. 
% \textbf{9.2.2} EDA (Easy Data Augmentation)

\subsection*{9.1 Back translation}
Back translation is a method to augment data by translating the text to another language and then translating it back to the original language. This method is used to generate new sentences that are similar to the original sentences. \\
Based on the research paper, \href{https://aclanthology.org/2023.findings-acl.518.pdf}{\textit{An Extensive Exploration of Back-Translation in 60 languages}} (Namee et al., 2023), The sentences are evaulated using BLEU score that ranges from $[0, 100]$. A higher BLEU score indicates the new sentence is more similar to the original in terms of vocabulary, grammar and fluency. \\
I will select languages that have a large change in BLEU score after back translation as this would mean the new sentence obtained is syntactically different from the original. \\
\newline
Firstly, the raw MITRE ATT\&CK description will be subjected to basic text cleaning such as removing weblinks, non-english words etc. before passing the text for translation. \\
Then the translated text will be converted back into English. The following 2 pages shows text samples before and after back translation; (sentences are colour coded for clarity). \\

\begin{landscape}
    \vspace*{\fill}
    \begin{table}[htbp]
        \caption{Back translation (Afrikaans)}
        \label{crouch}
        \begin{tabularx}{\linewidth}{p{0.1\linewidth}X}
            \toprule
            \textbf{ } & \textbf{Text example T-T1592.001} \\
            \midrule
            \textbf{Before} &
            \textcolor{blue}{Hardware gather information about the victims host hardware that can be used during targeting}  Information about hardware infrastructure include a variety of details such as types and versions on specific hosts as well as the presence of additional components that might be indicative of added defensive protections  ex  card biometric readers  dedicated encryption hardware  etc    \textcolor{red}{gather this information in various ways  such as direct collection actions via  Active Scanning  ex  hostnames  server banners  user agent strings  or  Phishing for Information}   also compromise sites then include malicious content designed to collect host information from visitors  \textcolor{blue}{Information about the hardware infrastructure also be exposed to adversaries via online or other accessible data sets  ex  job postings  network maps  assessment reports  resumes  or purchase invoices}   Gathering this information reveal opportunities for other forms of reconnaissance  ex   Search Open Websites Domains   or  Search Open Technical Databases   establishing operational resources  ex   Develop Capabilities   or  Obtain Capabilities   and or initial access  ex   Compromise Hardware Supply Chain   or  Hardware Additions   scanners be used to look for patterns associated with malicious content designed to collect host hardware information from visitors \textcolor{red}{Much of this activity have a very high occurrence and associated false positive rate  as well as potentially taking place outside the visibility of the target organization  making detection difficult for defenders}  Detection efforts be focused on related stages of the lifecycle  such as during Initial Access \\
            \midrule
            \textbf{After} &
            \textcolor{blue}{Hardware collects information about the victim's host hardware that can be used during targeting.} Information about hardware infrastructure includes a variety of details such as types and versions on specific hosts as well as the presence of additional components that may indicate additional defensive protections eg. biometric readers dedicated encryption hardware, etc. \textcolor{red}{collect this information in various ways, such as direct collection actions via active scanning ex hostnames server banners user agent strings or phishing for information} also compromise sites then include malicious content designed to collect host information from visitors \textcolor{blue}{Information about the hardware infrastructure is also exposed to adversaries via online or other accessible data sets ex job postings network maps assessment reports resumes or purchase invoices} Collecting this information reveals opportunities for other forms of reconnaissance e.g. Search open websites domains or Search open technical databases establishing operational resources e.g. Develop capabilities or Acquire capabilities and or initial access e.g. Compromise hardware supply chain or hardware add-on scanners are used to look for patterns associated with malicious content designed to collect host hardware information from visitors \textcolor{red}{Much of this activity has a very high incidence and associated false positive rate as well as potentially occurring outside the visibility of the target organization, making detection difficult for defenders.} Detection efforts are focused on related stages of the life cycle such as during initial access \\
            \bottomrule
        \end{tabularx}
    \end{table}
    \vspace*{\fill}
\end{landscape}
\begin{landscape}
    \vspace*{\fill}
    \begin{table}[htbp]
        \caption*{Back translation (Bengali)}
        \label{crouch}
        \begin{tabularx}{\linewidth}{p{0.1\linewidth}X}
            \toprule
            \textbf{ } & \textbf{Text example T-T1550.004} \\
            \midrule
            \textbf{Before} &
            \textcolor{blue}{Web Session Cookie can use stolen session cookies to authenticate to web applications and services  This technique bypasses some multi factor authentication protocols since the session is already authenticated} Authentication cookies are commonly used in web applications  including cloud based services  after a user has authenticated to the service so credentials are not passed and re authentication does not need to occur as frequently  \textcolor{red}{Cookies are often valid for an extended period of time  even if the web application is not actively used}  After the cookie is obtained through  Steal Web Session Cookie   or  Web Cookies   the then import the cookie into a browser they control and is then able to use the site or application as the user for as long as the session cookie is active  \textcolor{blue}{Once logged into the site  an can access sensitive information  read email  or perform actions that the victim account has permissions to perform} There have been examples of malware targeting session cookies to bypass multi factor authentication s Monitor for anomalous access of websites and cloud based applications by the same user in different locations or by different s that do not match expected configurations \\ 
            \midrule
            \textbf{After} &
            \textcolor{blue}{Web session cookies can use stolen session cookies to authenticate to web applications and services.This technique bypasses some multi-factor authentication protocols since the session is already authenticated.} are not passed and re-authentication is not required because \textcolor{red}{cookies are often valid for an extended period of time even if the web application is not actively being used} after receiving the cookie via a still web session cookie or a web cookie that is then imported. Cookies in a browser they control and then as long as the session cookie is active the user is able to use the site or application once logged in to the site can \textcolor{blue}{access a sensitive information read email or perform actions that the victim account has permission to perform multi} There are examples of malware targeting session cookies to bypass factor authentication \\ 
            \bottomrule
        \end{tabularx}
    \end{table}
    \vspace*{\fill}
\end{landscape}

As shown, back translation can yield very different results. \\