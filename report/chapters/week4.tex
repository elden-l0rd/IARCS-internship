\section*{Week 4: Refining the corpus and model}
\addcontentsline{toc}{section}{Week 4: Refining the corpus and model}

Some minor improvements on keyword filtering is done like removing duplicate words. It can be found in \hyperref[subsec:appendix1]{Appendix 1}\\

\textbf{Creating a set of keywords using Microsoft's definition of STRIDE}

\textbf{S:} Involves illegally accessing and then using another user's authentication information, such as username and password \\\\
\textbf{T:} Involves the malicious modification of data. Examples include unauthorized changes made to persistent data, such as that held in a database, and the alteration of data as it flows between two computers over an open network, such as the Internet \\\\
\textbf{R:} Associated with users who deny performing an action without other parties having any way to prove otherwise—for example, a user performs an illegal operation in a system that lacks the ability to trace the prohibited operations. Non-Repudiation refers to the ability of a system to counter repudiation threats. For example, a user who purchases an item might have to sign for the item upon receipt. The vendor can then use the signed receipt as evidence that the user did receive the package \\\\
\textbf{I:} Involves the exposure of information to individuals who are not supposed to have access to it—for example, the ability of users to read a file that they were not granted access to, or the ability of an intruder to read data in transit between two computers \\\\
\textbf{D:} Denial of service (DoS) attacks deny service to valid users—for example, by making a Web server temporarily unavailable or unusable. You must protect against certain types of DoS threats simply to improve system availability and reliability \\\\
\textbf{E:} An unprivileged user gains privileged access and thereby has sufficient access to compromise or destroy the entire system. Elevation of privilege threats include those situations in which an attacker has effectively penetrated all system defenses and become part of the trusted system itself, a dangerous situation indeed \\\\

\textbf{Create keywords from the definitions}\\\\
\textbf{S:} ['authenticate', 'username', 'password', 'access'] \\\\
\textbf{T:} ['modify', 'persistent', 'database', 'alter', 'open', 'network', 'internet'] \\\\ % persistent data as in the data that is not meant to be modified and infrequently accessed
\textbf{R:} ['deny', 'action', 'prove', 'non-repudiation', 'item', 'sign', 'receipt', 'receive', 'evidence', 'package', 'untrace',] \\\\
\textbf{I:} ['exposure', 'individual', 'access', 'file', 'granted', 'intruder', 'transit'] \\\\
\textbf{D:} ['denial', 'service', 'dos', 'web', 'server', 'unavailable', 'unusable', 'system', 'available', 'reliable'] \\\\
\textbf{E:} ['unprivileged', 'privileged', 'access', 'compromise', 'entire', 'system', 'elevation', 'penetrate', 'defenses', 'untrusted', 'trusted'] \\\\

\textbf{Keywords obtained from manually filtering the corpus}\\\\
\textbf{S\_keep:} ['information', 'detection', 'take',  'include', 'malicious', 'control', 'network', 'search', 'name', 'access', 'infrastructure', 'traffic', 'data', 'suspicious', 'trust', 'reconnaissance', 'email', 'phishing', 'resource', 'initial', 'visibility', 'monitor', 'server', 'form', 'open', 'potentially', 'websites', 'address', 'process', 'detect', 'credential', 'file', 'certificate', 'internet', 'install', 'key', 'online', 'link', 'source'] \\\\
\textbf{T\_keep:} ['malicious', 'file', 'activity', 'process', 'execute', 'access', 'information', 'control', 'software', 'modify', 'network', 'data', 'abuse', 'exe', 'manipulate', 'bypass', 'malware', 'functionality', 'integrity', 'dll', 'anomaly', 'install'] \\\\
\textbf{R\_keep:} ['user', 'application', 'api', 'activity', 'audit', 'source', 'system', 'native', 'hide', 'error', 'intrusion', 'function', 'record', 'clear', 'gcp', 'permission', 'analysis', 'collection', 'updatesink', 'indicate', 'detection', 'data', 'collect', 'environment', 'call', 'limit', 'cloudtrail', 'loss', 'conduct', 'prior', 'delete', 'cloud', 'configservicev', 'cloudwatch', 'diagnostic', 'capability', 'sufficient', 'insight', 'avoid'] \\\\
\textbf{I\_keep:} ['data', 'network', 'activity', 'access', 'behavior', 'environment', 'process', 'detection', 'remote', 'base', 'target', 'tool', 'file', 'api', 'traffic', 'acquire', 'application', 'host', 'infrastructure', 'device'] \\\\
\textbf{D\_keep:} ['service', 'target', 'tool', 'command', 'cause', 'server', 'network', 'outside', 'denial', 'dos', 'availability', 'high', 'destruction', 'infrastructure'] \\\\
\textbf{E\_keep:} ['process', 'access', 'file', 'execute', 'activity', 'execution', 'network', 'behavior', 'create', 'control', 'log', 'privilege', 'application', 'service', 'within', 'event', 'account', 'modify', 'run', 'abuse', 'monitoring', 'environment', 'binary', 'credential', 'enable', 'api', 'exe', 'function', 'payload', 'target', 'method', 'services', 'launch', 'root', 'os', 'many''accounts'] \\

\(\rightarrow\) Now, merge both lists together by group to get for example \textit{S\_final}. \\

Summary: Next week, to compare the semantic similarity of keywords to choose the final set of keywords to use for training the model. If it does not work, I will look into the Large Language Model Meta AI \textit{LLaMA} language model.